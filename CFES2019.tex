\documentclass[12pt,english]{scrartcl}
\usepackage[T1]{fontenc}
\usepackage{color}
\usepackage{array}
\usepackage{url}
\usepackage{pdfpages}

\usepackage[utf8]{inputenc}
\usepackage[english]{babel}
\usepackage{csquotes}
%\usepackage[backend=biber, style=draft]{biblatex}
\usepackage[backend=biber]{biblatex}
\addbibresource{MooreAfter2017.bib}
\addbibresource{inat.bib}
\addbibresource{misc.bib}
\usepackage[breaklinks=true, colorlinks=True, allcolors=blue]{hyperref}

\usepackage{indentfirst} 
\usepackage{comment}

% A couple of very simple macros to add 'Activities' and 'Plans' headings.
\newcommand{\activities}{\medskip\textbf{Activities}}
\newcommand{\plans}{\medskip\textbf{Plans}}

\makeatletter

\makeatother

\begin{document}

\title{CFES Report 2019}

\author{Aubrey Moore, Ph.D.\\
Associate Professor / Extension Entomologist}

\maketitle

\includepdf[pages=1]{Reflective2019-form}

\tableofcontents{}

\clearpage

\section{Preface}
\begin{refsection}
	
I was hired by the University of Guam on October 1, 2003 under a limited-term,
split appointment (50\% extension and 50\% research). On June 26,
2008, I started a tenure-track appointment as extension entomologist
(100\% extension) with the academic rank of assistant professor. At
the end of the 2012 fall term I applied for tenure and promotion to associate professor and
received both in 2013. At the end of 2018 fall term I applied for promotion to
full professor and was promoted on July 11, 2019~\cite{recommendation_for_promotion2019}. 

I work within the Agriculture and Natural Resources Unit of the University
of Guam Cooperative Extension Service. I am a faculty member of the
Environmental Science Graduate Program and a member of the Western
Pacific Tropical Research Center. 

This report documents my activities from June 15, 2018 through the present.

My current faculty role allocation is as follows:
\begin{itemize}
	\item 51\% Extension and Community Activities 
	\item 34\% Creative/Scholarly Activity or Research 
	\item 15\% University and Community Service
\end{itemize}

\textbf{Note to Reader:}

This report is available as an electronic document in PDF format at\\
\url{https://github.com/aubreymoore/CFES2019/blob/master/CFES2019.pdf}. 

If you are reading the PDF version of the report on a device connected
to the internet, you will be able to follow hypertext links to documents
I have referenced.

\printbibliography

\end{refsection} 

\pagebreak

\section{Extension and Community Activities}

\subsection{Insect Diagnostic Services}
\begin{refsection}
	
As an extension entomologist, a major part of my job is providing
insect identification and pest control recommendations to a diverse
clientele including commercial growers, gardeners, householders, GovGuam
agencies, federal agencies, and UOG colleagues. Most client contacts
are initiated by a phone call or a visit by the client to the ANR
office. In many cases identification and pest control recommendations
require a site visit by me and/or extension associates to collect
samples and define the problem.

\activities

\begin{itemize}
\item The number of extension calls requiring my assistance averaged approximately
three per day during the reporting period. Many of these are documented
as postings to iNaturalist \cite{moore_aubrey_2020}.
\end{itemize}

\plans

\begin{itemize}
\item I plan to continue providing insect diagnostic services.
\end{itemize}

\printbibliography

\end{refsection}


\subsection{Detection and Documentation of Invasive Species}
\begin{refsection}

Invasive insects are arriving on Guam at a very high rate (estimates
range as high as one new species per day). Very few of these are detected
and even fewer are identified because Guam suffers from \href{https://en.wikipedia.org/wiki/Taxonomic_impediment}{the taxonomic impediment}.
Even when reliable species determinations are made, new island records
are only rarely documented in the scientific press. Thus, impacts
of invasive insects on Guam and elsewhere in Micronesia are grossly
underestimated. One of my professional goals is to work towards solving
this problem by increasing the detection rate, getting specimens identified
by qualified taxonomists, and publishing new island records in the
scientific literature.

\activities

\begin{itemize}
	
\item Ten new island records for insects in Micronesia were documented in iNaturalist posts during the reporting period \cite{inat36470788,inat36285968,inat35845152,inat32572967,inat31326484,inat29333274,inat18166461,inat16734728,inat15747194,inat15067449,inat13466275}.

\item New Larval Host Record: \textit{Traminda aventiaria} (Lepidoptera: Geometridae) Feeds on the Critically Endangered Tree, \textit{Serianthes nelsonii} (Leguminosae), on Guam~\cite{moore_new_2020}

\end{itemize}

\plans

\begin{itemize}

\item The International Union for Conservation of Nature (IUCN-ISSG) is
building a Global Register of Introduced and Invasive Species. I have
volunteered to coordinate building a check list for species on Guam.

\item The Guam Invasive Species Council is required to maintain a list on
invasive species on Guam. I have volunteered to be ``registrar''
for this list.

\end{itemize}

\printbibliography
\end{refsection}

\subsection{University of Guam Insect Collection}
\begin{refsection}

The UOG insect collection is a valuable reference collection for extension
entomology, teaching and research. I am a member of the board of directors
for the collection and I work with Dr. Ross Miller to curate and catalog
this collection.

\activities

\begin{itemize}

\item I ported the digital catalog for the UOG Insect Collection from a
CSIRO BioLink database to a more modern web-based Symbiota database
which is now online \cite{moore_scan_2018}.

\item I established an internship to train entomology students how to curate
an institutional insect collection \cite{moore_internship_2018}.

\item The Benita Laird-Hopkins collection includes more than 5,000 insect
specimens reared from seeds of forest plants from Saipan and Guam
as part of the Ecology of Bird Loss Project. This collection has been
cataloged and accessioned into the UOG insect collection and a publication
is being prepared \cite{laird-hopkins_[preparation]_2018}.

\item In June 2018, I attended the Second Annual Digital Data in Biodiversity
Research Conference sponsored by iDigBio (Integrated Digital Biocollections)
to attend a workshop entitled Sharing and Mobilization of Massive
Specimen Image Databases from Collections of Tropical Island Biodiversity
as an invited participant. I made a presentation on building a biodiversity
inventory for Guam \cite{moore_building_2018-1} and discussed ongoing
collaboration with Dr. Alex Vandam on writing an NSF proposal to support
digitization of biological collections on American-affiliated islands
\cite{moore_trip_2018}.
\end{itemize}

\plans

\printbibliography

\end{refsection}

\subsection{Guam Coconut Rhinoceros Beetle Project}

This is my largest and most important project. Please see CRB activities in the Creative/Research/Scholarly section~\ref{sec:Coconut-Rhinoceros-Beetle}.

\subsection{National Plant Diagnostic Network (NPDN)}
\begin{refsection}
	
I serve as the UOG Coordinator for the National Plant Diagnostic Network. 

\activities

\begin{itemize}
\item Participated in monthly conference calls.
\item Prepared an annual work plan and budget \cite{moore_university_2018}.
\item Prepared annual report \cite{moore_npdn_2018}.
\item Served on the NPDN IT Strategic Planning Committee.
\item Trained and certified 14 First Detectors as part of my AL/BI 345 General
Entomology course.
\end{itemize}

\plans

attend meeting in AZ
PIDDRS

\printbibliography

\end{refsection}	

\subsection{Guam Invasive Species Advisory Committee (GISAC) and Guam Invasive Species Council (GISC)}
\begin{refsection}
	
I am a founding member and regular participant in GISAC. President Underwood delegated me to represent UOG as a voting member of GISC.

\activities

\begin{itemize}
	
\item I participated in GISAC and GISC meetings.

\item During 2018, I served on a GISC Import Data Harmonization Committee.
This committee generated recommendations \cite{guerrero_guam_2018}
resulting in a bill to amend the Guam Invasive Species Act \cite{guerrero_bill_2018}.

\end{itemize}

\plans

\begin{itemize}
	
\item I plan to continue as an active member of GISAC and GISC.

\item I plan to participate in a review of the Guam Invasive Speices Management Plan.

\end{itemize}

\printbibliography

\end{refsection}

\subsection{Public Outreach (Guest lectures, presentations, interviews)}
%\begin{refsection}
	
\subsubsection{Guest Lectures}

\subsubsection{Presentations}
\begin{refsection}
	
\nocite{*}
\defbibfilter{presentations}{keyword=presentation2018 or keyword=presentation}
\printbibliography[filter=presentations]
\end{refsection} 

\subsubsection{Interviews}


%\end{refsection}

\begin{comment}

\raggedright\vspace{2mm}\textbf{Activity}
\begin{itemize}
\item Presentations \cite{moore2018building2,blas2018protecting,deloso2018parasitoid,moore2018freecell,moore2018coconut,moore2018biological2,moore2018freecell2,marshall2018progress,moore2018attempted,moore2017impactof,moore2018biological,moore2018building,moore2017invasion,moore2017coconut,moore2017usingfree,moore2017accessto,moore2017biological,moore2017thecoconut,moore2017biological2,moore2017biological3}
\item Workshops \cite{berringer2018sixteenth,moore2017bringyour,moore2018cnasworkshop,quitugua20182018coconut}
\item Press \cite{moore2018special,varietyuogseeks,pacific2018scientists,2018viralcontrol,postnewtree,leonguerrero2018interview,2018g2ghuman,2017tracking}
\end{itemize}
\raggedright\vspace{2mm}\textbf{Reference(s)}

\begin{btSect}[vancouver]{zotero}
\btPrintCited
\end{btSect}
\newpage{}
\end{btUnit}

\begin{btUnit}

\end{comment}

\subsection{Public Outreach(Internet)}

\begin{comment}

\raggedright\vspace{2mm}\textbf{Activity}
\begin{itemize}
\item On-line output \cite{moore2018onlinecatalog,moore2018textitcitripestis2,moore2018textitcitripestis,moore2018lobatelac2,moore2018scanuniversity,moore2017website,moore2018insectpin,moore2018checklist,moore2018citripestis,moore2018lobatelac,manuel2017pacific,moore2018inaturalist,moore2018interactive,moore2017listof,moore2017guamforestry,moore2017crbgarticlereview,moore2017crbtrap,moore2017techblog,moore2017scaevoladieback,moore2017failedattempt,moore2017publish,moore2017usingscrapy,moore2017tweeking,moore2017install,moore2017setting,moore2017usingthe,moore2017finding,moore2017usingscrapy2,moore2017calculate,moore2017converting,moore2017migrate,moore2018croppestlist}
\end{itemize}
\raggedright\vspace{2mm}\textbf{Reference(s)}

\begin{btSect}[vancouver]{zotero}
\btPrintCited
\end{btSect}
\newpage{}
\end{comment}

\pagebreak

\section{Creative/Scholarly Activities or Research}

\subsection{\label{sec:Coconut-Rhinoceros-Beetle}Coconut Rhinoceros Beetle (CRB)
Biocontrol}

This is my largest and most important project. Funding for outreach
and applied research is currently provided by three grants: USDA-APHIS
FY17 Farm Bill, USDA-Farm FY18 Bill, and a grant from the Department
of the Interior-Office of Island Affairs for FY18-19.

I have submitted a proposal for FY19 Farm Bill Fundings. The abstract
from this proposal serves as a description of this ongoing project:

A newly discovered biotype of coconut rhinoceros beetle (CRB-G) is
rapidly killing coconuts and other palms on Guam and on other Pacific
islands. Following a failed eradication attempt on Guam, CRB-G proved
hard to control because it is resistant to \emph{Oryctes rhinoceros}
nudivirus (OrNV), which was previously used as the preferred biological
control agent for control of CRB outbreaks on Pacific Islands and
elsewhere. Previous to the discovery of CRB-G, all OrNV releases on
Pacific Islands resulted in immediate and sustained suppression of
CRB damage to low levels and prevented tree mortality.

Guam is currently experiencing an uncontrolled and unmonitored island-wide
CRB-G outbreak which was triggered by abundant CRB-G breeding sites
in the form of dead and dying vegetation left in the wake of Typhoon
Dolphin which occured in May 2015. of a recent typhoon. Most of these
breeding sites are inaccessable to sanitation efforts, being either
in the jungle or on military land (which covers one third of Guam).
A positive feedback cycle has begun whereby large numbers of adult
beetles are killing large numbers of palms which become breeding sites
which generate even higher numbers of adults. Severe damage to Guam\textquoteright s
palms prompted the Governor of Guam to declared a state of emergency
in July 2017.

The main objective of this project is to stop the uncontrolled outbreak
on Guam. Entomologists working on the CRB-G problem on several Pacific
islands agree that the most feasible tactic to halt tree mortality
and suppress damage to tolerable levels is establishment of biological
control using an isolate of OrNV which is highly effective as a biological
control agent for CRB-G. We are working with collaborators to identify
populations of CRB-G throughout the Asia-Pacific region. We will sample
these populations for biological control agent candidates which will
be evaluated in laboratory bioassays performed at UOG. Promising candidates
will be field released using autodissemmination as per a USDA-APHIS
import and release permit.

Concurrent with establishment of CRB-G biocontrol, success of the
project will be monitored in a quarterly, island-wide tree health
survey and incidence of OrNV infection will be monitored in a subsample
of all field collected CRB-G.

If the Guam CRB-G infestation cannot be controlled, it is expected
that most palms on the island will be killed and CRB-G will continue
to spread to other islands and beyond. If CRB-G invades smaller islands
and atolls where coconut is the tree of life, a human tragedy will
ensue. On larger islands, coconut and oil palm industries will be
severely impacted. Attempts to organize a regional project in response
to CRB-G are underway.

\begin{comment}

\raggedright\vspace{2mm}\textbf{Activity}
\begin{itemize}
\item Coauthored a peer-reviewed journal article documenting discovery of
CRB-G \cite{marshall2017anew}.
\item Wrote a magazine article for the Guam Invasive Species Awareness week.
This was published by the Pacific Islands Times \cite{moore2018special}.
A similar article was archived in Zenodo \cite{aubreymoore2018theguam}.
\item Recruited Dr. James Grasela, an insect pathologist, to work on the
project for two years using funding from the US Department of Interior
- Office of Island Affairs. Grasela's itinital task will be to perform
laboratory biassays to evaluate OrNV isolates as candidates for biocontrol
of CRB-G (Job announcement: \cite{moore2018position}).
\item Recruited Ian Iriarte as a research assistant using funds from Farm
Bill grants. Ian is also my graduate student. He is working with me
on development of an automated coconut rhinoceros beetle damage monitoring
system using computer vision and deep learning. This project is likely
to be the topic of his master's thesis.
\item In August 2018, Moore, Grasela, Iriarte and Quitugua participated
in the 51st Annual Meeting of the Society for Invertebrate Pathology
and International Congress on Invertebrate Pathology and Microbial
Control held at the Gold Coast, Australia. This conference provided
a venue for was a symposium and a meeting to plan and promote collaboration
among Pacific entomologists working on the CRB-G problem \cite{moore2018failedattempts,marshall2018progress}.
\item Created a private wiki site to facilitate sharing scientific/technical
information among scientists working on the CRB-G problem \cite{moore2018crbgwiki}.
\item Laboratory bioassays of an OrNV isolate propagated from a virus-infected
CRB-G adult we collected on Negros Island, Philippines in 2017 produced
no response when applied to CRB-G adults \cite{moore2018initial}
\end{itemize}
\raggedright\vspace{2mm}\textbf{Reference(s)}

\begin{btSect}[vancouver]{zotero}
\btPrintCited
\end{btSect}
\newpage{}
\end{btUnit}

\begin{btUnit}

\end{comment}

\subsection{Cycad Aulacaspis Scale (CAS) Biocontrol}

A US Forest Service survey published in 2002 reported that the most
abundant tree in Guam's forests (DBH > 5 inches) was Guam's endemic
cycad, \emph{Cycas micronesica}. In 2003, an invasive scale insect,
\emph{Aulacaspis yasumatsui,} was detected on ornamental cycads but
it soon infested wild cycads and started killing them. Within a decade,
90\% of Guam\textquoteright s endemic cycads have been killed by the
scale and other invasive species. \emph{Cycas micronesica} was placed
on the US National Endangered Species List in 2015.

Mature plants are protected by a lady beetle I introduced, but no
natural reproduction is occurring because seeds and seedlings are
still being killed by the scale insect. A likely solution to this
problem is establishment of a small biocontrol agent, such as a miniature
parasitic wasp which will control scale insects infesting seeds and
seedlings.

\begin{comment}

\raggedright\vspace{2mm}\textbf{Activity}
\begin{itemize}
\item Worked with Ben DeLoso, Tom Marler's grad student, to perform a CAS
parasitoid survey \cite{deloso2018parasitoid}. 
\end{itemize}
\raggedright\vspace{2mm}\textbf{Reference(s)}

\begin{btSect}[vancouver]{zotero}
\btPrintCited
\end{btSect}
\newpage{}
\end{btUnit}

\begin{btUnit}

\end{comment}

\subsection{Guam Forest Insect Survey}

The objective of this project is to compile a comprehensive check
list of insects impacting Guam's forests. While it is notable that
Guam's two most numerous forest trees, namely fadang, \emph{Cycas
micronesica}, and coconut palm, \emph{Cocos nucifera}, are under simultaneous
attack by invasive insects, there are many other forest plants under
attack from invasive insects. This project is funded by McIntire-Stennis.

\begin{comment}
\raggedright\vspace{2mm}\textbf{Activity}
\begin{itemize}
\item I work closely with Jim McConnell's Guam Plant Extinction Prevention
Program. Many of Guam's rare plants are being attacked by invasive
insects. I routinely identify and document insect specimens collected
from the GPEPP plant nursery and from field surveys.
\item Annual report \cite{moore2018mcintirestennis2}.
\item Proposal \cite{moore2018mcintirestennis}.
\end{itemize}
\raggedright\vspace{2mm}\textbf{Reference(s)}

\begin{btSect}[vancouver]{zotero}
\btPrintCited
\end{btSect}
\newpage{}
\end{btUnit}

\begin{btUnit}
\end{comment}

\subsection{Eight Spot Butterfly (ESB) Conservation}

The Guam Department of Agriculture Division of Aquatic and Wildlife
Resources (GDOA-DAWR) requested assistance with conservation of the
rare Mariana eight-spot butterfly, \emph{Hypolimnas octocula marianensis.
}I grant proposal for this work was funded by US Fish and Wildlife
and funds were made available to the Guam Department of Agriculture
for this work. The project was halted shortly after it began because
USFWS listed ESB on the National Endangered Species List. This required
a permit to work with this species. I worked with GDOA-DAWR on a permit
application. I am ready to restart this project, but GDOA-DAWR is
unable to access grant funding from GovGuam.

\begin{comment}
\raggedright\vspace{2mm}\textbf{Activity}
\begin{itemize}
\item Progress on this project blocked by GovGuam beaurocracy. No progress
to report. \newpage{}
\end{itemize}
\end{btUnit}

\begin{btUnit}
\end{comment}

\subsection{Guam Biodiversity Inventory}

I consider this to be my second most important project.

A biodiversity inventory is essentially a database containing a comprehensive
check list of all taxa known occur within a defined area.

A terrestrial biodiversity inventory for Guam is needed to document
rapid changes to Guam\textquoteright s ecosystems, to provide free
and open access to information on Guam\textquoteright s flora and
fauna, and to share Guam biodiversity information with the global
scientific community, policy makers and the public.

The Guam Biodiversity Inventory will facilitate automatic generation
and updates to lists such as: a list of all invasive species on Guam
with year first recorded, a list of new species described from specimens
collected on Guam, a list of observations for Guam\textquoteright s
endangered species, a list of Guam\textquoteright s native plants
with associated herbivores and pathogens, and a list of crops grown
on Guam and pests and pathogens which attack them.

\begin{comment}
\raggedright\vspace{2mm}\textbf{Activity}
\begin{itemize}
\item I made a couple of presentations on my plans for the Guam Biodiversity
Inventory \cite{moore2018building2,moore2018building}.
\item I designed data model for the Guam Biodiversity Inventory and created
a prototype web site.
\end{itemize}
\raggedright\vspace{2mm}\textbf{Reference(s)}

\begin{btSect}[vancouver]{zotero}
\btPrintCited
\end{btSect}
\end{comment}

\pagebreak

\section{University and Community Service}

\subsection{Instruction}

\begin{comment}
\raggedright\vspace{2mm}\textbf{Activity}
\begin{itemize}
\item According to the UOG Registrar, I taught 4 courses during the Fanuchanan
semester: AL345, AL345L, BI345, and BI345L. In reality this was a
single course, AL/BI 345 \emph{General Entomology} consisting of two,
one and a half hour lectures and one three hour lab per week. 
\begin{itemize}
\item I built and maintained a web site for this course \cite{moore2017website}
\item In student evaluations for AL345, AL345L, BI345, and BI345L, my scores
were consistently higher than the University and College average \cite{moore2018student}.
\end{itemize}
\item I acted as the major faculty advisor for Mr. Ian Iriarte who is pursuing
a Master's degree in Environmental Science.
\end{itemize}
\raggedright\vspace{2mm}\textbf{Reference(s)}

\begin{btSect}[vancouver]{zotero}
\btPrintCited
\end{btSect}
\newpage{}
\end{btUnit}

\begin{btUnit}
\end{comment}

\subsection{Faculty Committees}

\subsubsection{Faculty Building Facilities Committee for the ALS}

This committee was formed by the Agriculture and Life Sciences Division
to provide advice to the Dean on facilities problems within the Agriculture
and Life Sciences Building. During the reporting period, I was re-elected
as chair of this committee and I am joined by Dr. Jim McConnell and
Dr. LaJoy Spears as the other members.

\raggedright\vspace{2mm}\textbf{Activity}
\begin{itemize}
\item Plans for improvements to the ALS124 teaching lab have been only partially
achieved. For the past three years, faculty have asked for a dedicated
computer and modern audiovisual equipment to facilitate science teaching.
During the reporting period, lab tables were equipped with power sockets
to replace those removed during a previous renovation.
\item We continue to struggle with finding solutions to chronic air conditioning
problems.
\end{itemize}

\subsubsection{Search Committee: Extension Animal Scientist}

I chair this committee. I am joined by Mari Marutani, LaJoy Spears,
Bob Schlub, and Tom Poole, Guam's Territorial Veterinarian.

\begin{comment}

\raggedright\vspace{2mm}\textbf{Activity}
\begin{itemize}
\item Position announcement written \cite{moore2018uoganimal} and advertisment
placed on the web site of the American Association of Animal Scientists
\cite{moore2018animalscientist}.
\end{itemize}

\end{comment}

\subsubsection{Search Committee: Extension Agricultural Economist}

I am a member of this committee and I am joined by Bob Barber (chair),
LaJoy Speers, and John Brown.

\subsubsection{Search Committee: Research Associate II (CRB Project)}

I chaired this committee and was joined by Jim Grasela, Roland Quitugua,
and Jesse Bamba.

\subsubsection{Continuing Employment Committee: Austin Shelton}

I chair this committee and I am joined by Ross Miller and Hui Gong.

\subsubsection{Continuing Employment Committee: Andrea Blas}

I served on this committee with Ross Miller and Frank Camacho.

\subsubsection{Extension Publications Committee}

I served as a member of this committee.

\begin{comment}
\raggedright\vspace{2mm}\textbf{Reference(s)}

\begin{btSect}[vancouver]{zotero}
\btPrintCited
\end{btSect}
\end{comment}

\appendix

\section{Grants}

\begin{comment}
\begin{tabular}{>{\raggedright}p{2in}>{\raggedright}p{2in}>{\centering}p{0.5in}>{\raggedleft}p{1in}}
\hline 
\textbf{Funding Source} & \textbf{Title} & \textbf{Years} & \textbf{Budget}\tabularnewline
\hline 
\textbf{Active} &  &  & \tabularnewline
\hline 
Farm Bill 2017 \cite{moore2017farmbill} & Biological Control of Coconut Rhinoceros Beetle Biotype-G & 1 & \$200,000\tabularnewline
\hline 
Farm Bill 2018 \cite{moore2018farmbill} & Biological Control of Coconut Rhinoceros Beetle Biotype-G & 1 & \$200,000\tabularnewline
\hline 
Department of the Interior - Office of Insular Affairs

\cite{moore2017doiproposal} & Biological Control of Coconut Rhinoceros Beetle Biotype-G in Micronesia & 2 & \$176,553\tabularnewline
\hline 
McIntire-Stennis \cite{moore2018mcintirestennis2}  & Guam Forest Insect Survey & 4 & \$40,000\tabularnewline
\hline 
National Plant Diagnostic Network 2017 \cite{moore2018university}
\cite{moore2018npdnaccomplishments} &  & 1 & \$10,000\tabularnewline
\hline 
\textbf{Pending} &  &  & \tabularnewline
\hline 
Farm Bill 2019

\cite{moore2018fy19farm,moore2018fy19farm2} & Biological Control of Coconut Rhinoceros Beetle Biotype-G & 1 & \$282,044\tabularnewline
\hline 
McIntire-Stennis \cite{moore2018mcintirestennis} &  & 5 & \$80,000\tabularnewline
\hline 
National Plant Diagnostic Network &  & 1 & \$10,000\tabularnewline
\hline 
\end{tabular}

\end{comment}

\section{Selected Examples of My Work}

\begin{comment}
\begin{description}
\item [{Peer-reviewed~journal~article}] Sean D. G. Marshall, Aubrey Moore,
Maclean Vaqalo, Alasdair Noble, and Trevor A. Jackson, \textquotedbl A
new haplotype of the coconut rhinoceros beetle, \emph{Oryctes rhinoceros},
has escaped biological control by \emph{Oryctes rhinoceros} nudivirus
and is invading Pacific Islands'', Journal of Invertebrate Pathology
149 (2017), pp. 127-{}-134.\cite{marshall2017anew}
\item [{Peer-reviewed~journal~article}] Jake Manuel, W. John Tennent,
Donald W. Buden, and Aubrey Moore, \textquotedbl First record of
\emph{Doleschallia tongana} (Lepidoptera: Nymphalidae) for Guam Island\textquotedbl ,
F1000Research 7 (2018), pp. 366.\cite{manuel2018firstrecord}
\item [{Grant~proposal}] Aubrey Moore, \textquotedbl DOI Proposal: Biological
Control of Coconut Rhinoceros Beetle Biotype G in Micronesia\textquotedbl{}
(2017).\cite{moore2017doiproposal}
\item [{Grant~proposal}] Aubrey Moore, \textquotedbl McIntire-Stennis
Proposal: Guam Forest Biodiversity Inventory\textquotedbl{} (2018).\cite{moore2018mcintirestennis}
\item [{Magazine~article}] Aubrey Moore, \textquotedbl Special Report
for Guam Invasive Species Awareness Week: Invasive Species are a Crisis
for Guam and the Pacific, Right Now\textquotedbl , Pacific Island
Times (2018).\cite{moore2018special}
\item [{Oral~presentation~slide~set}] Aubrey Moore, \textquotedbl Building
a Terrestrial Biodiversity Inventory for Guam\textquotedbl{} (2018).\cite{moore2018building}
\item [{Web~site}] \cite{moore2018crbgwiki} Aubrey Moore, \textquotedbl CRB-G
Wiki - CRB-G Wiki\textquotedbl{} (2018).
\end{description}
\raggedright\vspace{2mm}\textbf{Reference(s)}\begin{btSect}[vancouver]{zotero}
\btPrintCited
\end{btSect}
\end{btUnit}

\end{comment}

\clearpage
\begin{refsection}
\nocite{*}
\printbibliography[]
\end{refsection}

\end{document}
